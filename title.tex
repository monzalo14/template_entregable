
%%%%%%%%%%%%%%%%%%%%%%%%%%%%%%%%%%%%%%%%%%%%%%%%%%%%%%%%%%%%%%%%%%%%%%%%%%%%%%%%%%
% First Page
%%%%%%%%%%%%%%%%%%%%%%%%%%%%%%%%%%%%%%%%%%%%%%%%%%%%%%%%%%%%%%%%%%%%%%%%%%%%%%%%%%
\graphicspath{{images/}}

\newcommand{\MyName}{Mónica Zamudio López}
\newcommand{\Institution}{Laboratorio de Datos, SEDESOL}
\newcommand{\DelTitle}{Recolección y limpieza de información}
\newcommand{\DelNumber}{2}
\newcommand{\DelVersion}{0.1/1.0}
\newcommand{\Contrato}{ATN/OC 15822-RG}
\newcommand{\footertext}{\raisebox{3mm}{Entregable \DelNumber}}
\setlength{\footheight}{36pt}
\newcommand{\footerlogo}{\raisebox{3mm}{\leavevmode\includegraphics[width=2cm]{bid}}}
\clearscrheadfoot
\pagestyle{empty}


% \begin{tikzpicture}[overlay,remember picture]
%   \draw [line width=1pt]
%   ($ (current page.north west) + (1cm,-1cm) $)
%   rectangle
%   ($ (current page.south east) + (-1cm,1cm) $);
% \end{tikzpicture}

\definecolor{SINetblue}{HTML}{07505B}
\newcolumntype{C}{ >{\centering\arraybackslash} m{4cm} }

\begin{center}
SEDESOL 2018
\vspace{0.1cm}

  \begin{center}


  % H2020 has no logo and no visual identity

  % \includegraphics[width=0.7\textwidth]{images/LOGO_FJR}
      \Large \MyName \\\vspace{5mm}
\begin{multicols}{2}
\includegraphics[width=0.35\textwidth]{images/bid}
\includegraphics[width=0.25\textwidth]{images/LOGO_FJR}
\end{multicols}
\begin{multicols}{2}
\includegraphics[width=0.4\textwidth]{images/sedesol}
\includegraphics[width=0.2\textwidth]{images/presidencia}
\end{multicols}


  \vspace{2mm}

  \end{center}
  \vspace{0.3cm}
  {\Large Título del proyecto: USO DE DATOS MASIVOS PARA LA EFICIENCIA DEL ESTADO Y LA INTEGRACIÓN REGIONAL\\}
    {\large Clave: \Contrato}\\
  \vspace{0.5cm}
  \Large Puesto: Científico de Datos Junior\\
  \vspace{1.0cm}

  \begin{spacing}{2.5}
    \textbf{\Huge \DelTitle}\\\vspace{10mm}
    \textbf{\Large Entregable número: \DelNumber} \\\vspace{10mm}
  \end{spacing}

  % \vspace*{\fill}

  %just to avoid warning :)
  \newcommand\undefcolumntype[1]{\expandafter\let\csname NC@find@#1\endcsname\relax}
  \newcommand\forcenewcolumntype[1]{\undefcolumntype{#1}\newcolumntype{#1}}
  \forcenewcolumntype{C}{ >{\arraybackslash} m{3cm} }


  % \begin{tabular}{C@{\hspace*{0cm}}l}
  %   % \includegraphics[scale=0.2]{images/logos/EU_Flag_320_213} &
  %   % \begin{tabular}{l}
  %   % {Funded by the European Union’s Horizon 2020 research and innovation programme}\\
  %   % {under the Marie Sklodowska-Curie Grant Agreement No. 699924}\\
  %   \end{tabular}
  % \end{tabular}
\end{center}

\clearpage

%%%%%%%%%%%%%%%%%%%%%%%%%%%%%%%%%%%%%%%%%%%%%%%%%%%%%%%%%%%%%%%%%%%%%%%%%%%%%%%%%%
% Second Page
%%%%%%%%%%%%%%%%%%%%%%%%%%%%%%%%%%%%%%%%%%%%%%%%%%%%%%%%%%%%%%%%%%%%%%%%%%%%%%%%%%
\setlength{\headheight}{0.7cm}
\setlength{\footskip}{18mm}
\addtolength{\textheight}{-\footskip}
\pagestyle{empty}

\begin{flushright}
    \begin{tabular}{lp{11cm}}
    \textbf{Acrónimo del proyecto:}       &   Estimación de Ingreso \\
    \hline
    \textbf{Nombre completo del proyecto:} & USO DE DATOS MASIVOS PARA LA EFICIENCIA DEL ESTADO Y LA INTEGRACIÓN REGIONAL\\
    \hline
    \textbf{Referencia:}                  &   ATN/OC 15822-RG\\
    \hline
%    \textbf{Topic:}                 &   ICT-10-2015 \\
%    \textbf{Type of Action:}        &   RIA \\
    % \textbf{Grant Number:}          &   699924 \\
    \textbf{URL del Proyecto:}           &  \url{http://www.plataformapreventiva.gob.mx}
  \end{tabular}



% define "struts", as suggested by Claudio Beccari in
%    a piece in TeX and TUG News, Vol. 2, 1993.
\newcommand\Tstrut{\rule{0pt}{2.6ex}}         % = `top' strut
\newcommand\Bstrut{\rule[-0.9ex]{0pt}{0pt}}   % = `bottom' strut

  \begin{tabular}{|l|p{115mm}|}\hline
    % \Tstrut\Bstrut Editor:& XXX, XXX-Institution\\\hline
    \Tstrut\Bstrut Tipo de Entregable:& Reporte (R) \\\hline
    % \Tstrut\Bstrut Dissemination level:& Public (PU)\\\hline
    \Tstrut\Bstrut Fecha de Entrega Contractual:& Agosto - 2018\\\hline
    \Tstrut\Bstrut Fecha de Entrega& 17 de agosto de 2018\\\hline
    % \Tstrut\Bstrut Suggested Readers:&Project partners, future community-lab.net users\\\hline
    \Tstrut\Bstrut Número de Páginas:&\pageref{finalpg}\\\hline
    \Tstrut\Bstrut Keywords:& estimación ingreso ciencia datos ingesta \\\hline
    \Tstrut Autor:&
    \begin{tabular}[t]{l}
      \MyName, \Institution \Bstrut \\
      % XXX - YYY, Institution \\
      % XXX - YYY, Institution \\
      % XXX - YYY, Institution \Bstrut \\
    \end{tabular}\\\hline
  \end{tabular}
\end{flushright}

\section*{Resumen}
Este proyecto nace de la intención de la Secretaría de Desarrollo Social de construir una metodología de focalización más eficiente y así distribuir mejor los recursos de los programas sociales. Para el desarrollo de esta metodología se tomó en cuenta el problema de sub y sobrerreportaje, considerando distintas especificaciones para construir redes bayesianas. Se utilizaron como principales fuentes de datos la Encuesta de Características Socioeconómicas de los Hogares (ENCASEH), su módulo de verificaciones domiciliarias, y los datos de la Encuesta Nacional de Ingreso y Gasto de los Hogares (ENIGH). \\
Una vez concluído el proceso de obtención e integración de estos datos, se procedió a la exploración y modelado. A la par, se comenzó a desarrollar la infraestructura para que el flujo de datos fuera reproducible hasta esta etapa, y a la vez las poblaciones descritas por los datos fueran lo más comparables posible. \\
En este documento se detalla parte del desarrollo que permitió el análisis posterior, así como los resultados de ese análisis.
\clearpage

