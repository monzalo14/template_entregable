\chapter{Fuentes de datos}

\section{CUAPS}
El \textbf{Cuestionario} \textbf{Único} \textbf{de} \textbf{Aplicación} \textbf{a} \textbf{Programas} es una herramienta utilizada por SEDESOL para obtener información de diseño y focalización de  los programas sociales. Para la integración de los datos, la DGGPB envía anualmente un formato en Excel a los responsables de programas. A partir de 2017, se ha buscado migrar gradualmente a un llenado a través de la plataforma de SISI, por lo que hoy se cuenta con información de un mismo cuestionario, llenado de dos maneras distintas. Los datos de CUAPS son entregados al laboratorio en tres tablas distintas:
\begin{itemize}
    \item \textbf{Programas} - Esta tabla contiene los datos de diseño a nivel programa. Se especifica información administrativa del programa, sus objetivos generales y específicos de diseño, y los derechos sociales que el programa busca atender.
    \item \textbf{Componentes} - Para operar la política social, los programas definen apoyos que responden a los objetivos definidos en su diseño. Estos apoyos son la unidad más granular de suministro de beneficios utilizada por los programas, y pueden estar clasificados dentro de “bloques temáticos” llamados componentes.\footnote{Como un ejemplo simple: un programa de Mejoramiento a la Vivienda puede tener componentes de pisos, muros y techos, y otorgar tres apoyos diferentes para la componente de muros: cemento, varillas y un subsidio al salario de trabajadores de la construcción.}
    \item \textbf{Focalización} - Esta tabla contiene información de la focalización de programas, a nivel criterio de focalización. Los programas pueden tener uno o más criterios de focalización para definir a su población objetivo, y estos criterios pueden estar lógicamente relacionados. A partir de la información de la tabla, es posible reconstruir esas matrices de diseño y delimitar la población objetivo de los apoyos otorgados por los programas.
\end{itemize}
\section{PUB}
El \textbf{Padrón Único de Beneficiarios} es el resultado de la consolidación de todos los padrones de los distintos programas sociales, conformado con el fin de monitorear los apoyos que reciben los beneficiarios. La actualización de los padrones es trimestral, y potencialmente corrige observaciones de trimestres anteriores al trimestre de actualización. En el PUB se tiene el nombre, sexo y grupo de edad del beneficiario, e información básica del programa del cual es beneficiario.
\section{CUIS-ENCASEH}
El Cuestionario Único de Información Socioeconómica es el instrumento mínimo\footnote{Los programas pueden hacer más preguntas de las que se incluyen en el CUIS, pero no menos.} utilizado por los programas para hacer levantamientos socioeconómicos, y es aplicado a nivel hogar. Cada programa gestiona los datos para su operación, y la información que corresponde al CUIS se va integrando a lo largo del tiempo por la DGGPB. Así, la dirección cuenta con datos históricos de los hogares que alguna vez han sido sujetos a levantamientos socioeconómicos.
\section{SIFODE}
El Sistema de Focalización de Desarrollo es una herramienta diseñada para coordinar la operación de los programas sociales, administrada dentro de la DGGPB. En esencia, su operación consiste en utilizar la información socioeconómica de los hogares con los que se cuenta para crear universos potenciales de personas beneficiarias de los programas sociales. Para esta tarea, se toman bloques temporales de datos del CUIS y del PUB, y se genera información sobre la situación de ingreso y carencias sociales enfrentada por los hogares encuestados. Estos elementos son los factores que componen la definición de pobreza multidimensional\footnote{Ver \ref{coneval_folleto} para más detalle} del Consejo Nacional de Evaluación de la Política de Desarrollo Social (CONEVAL):
\begin{itemize}
    \item Ingreso corriente per cápita - el ingreso compone la detección de pobreza y vulnerabilidad a través de dos puntos de corte: la Línea de Bienestar Económico\footnote{Esta especifica el ingreso necesario para adquirir las canastas alimentaria y no alimentaria de bienes y servicios. Ver \ref{coneval_medicion} para mayor detalle.} y la Línea de Bienestar Mínimo.\footnote{Esta especifica el ingreso necesario para adquirir la canasta alimentaria.}
    \item Rezago educativo - El indicador de rezago educativo toma como elementos el acceso a la educación obligatoria correspondiente que tienen los distintos integrantes de los hogares. La persona se considera como no carente sólo si está en edad escolar y asiste a la escuela o si de acuerdo con su edad ha concluido la primaria o secundaria, considerando la legislación aplicable.
    \item Acceso a los servicios de salud - El indicador de acceso a los servicios de salud toma en consideración que las personas cuenten con adscripción o derecho a recibir servicios médicos en alguna institución de salud, ya sea pública o privada. La persona se considera como no carente sólo si cuenta con adscripción o filiación directa o indirectamente a alguna institución de este tipo.
    \item Acceso a la seguridad social - El indicador de acceso a la seguridad social toma en cuenta el acceso de las personas a coberturas sociales mínimas que deben otorgarse a los trabajadores y a sus familias. La seguridad social puede definirse como \textit{''el conjunto de mecanismos diseñados para garantizar los medios de subsistencia de los individuos y sus familias ante eventualidades como accidentes, enfermedades, la vejez o el embarazo.''}\footnote{Ver \ref{coneval_folleto}.}
    \item Calidad y espacios de la vivienda - El indicador de calidad y espacios en la vivienda toma en consideración que la vivienda cuente materiales que garanticen la estabilidad y firmeza de los pisos, techos y muros, así como que las personas que la habitan no vivan en condiciones de hacinamiento.
    \item Acceso a los servicios básicos en la vivienda - El indicador de acceso a los servicios básicos en la vivienda considera las condiciones sanitarias en las que viven las personas habitantes de la vivienda. En particular, se considera que dichas personas carecen de servicios básicos en la vivienda si la vivienda no cuenta con agua entubada dentro del terreno, drenaje y electricidad o si el combustible usado para cocinar es carbón, y la vivienda no cuenta con chimenea.
    \item Acceso a la alimentación - El indicador de acceso a la alimentación es construido a partir de la Escala Mexicana de Seguridad Alimentaria, y toma en consideración si las integrantes de la vivienda tuvieron alimentación basada en poca variedad, dejaron de tomar al menos una de las tres comidas del día, comieron menos de lo que piensan debían comer, se quedaron sin comida o sintieron hambre pero no comieron.
    \item Grado de cohesión social - El CONEVAL define tres espacios conceptuales para medir el desarrollo social de los mexicanos: el bienestar, los derechos sociales y el contexto territorial. El primero es medido a través de los cortes antes mencionados al ingreso, el segundo por los indicadores de acceso mencionados, y el último por el grado de cohesión social en el espacio del territorio. Las propuestas retomadas por CONEVAL con respecto a su conformación comprenden la desigualdad económica, la razón de ingreso entre la población en pobreza extrema y la no pobre y no vulnerable, la polarización social y las redes sociales. Sin embargo, no existe un consenso sobre los indicadores a utilizar, así que el grado de cohesión social constituye una dimensión teórica de la pobreza multidimensional pero no de su medición actualmente.
\end{itemize}
Una vez obtenidos esos indicadores de bienestar y carencias, esta información es integrada con los datos del PUB, para así generar universos de beneficiarios potenciales de programas y entonces optimizar la frecuencia de los levantamientos socieconómicos.

