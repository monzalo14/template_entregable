\chapter{Introducción}
\section{Fuentes de Información}
La Secretaría de Desarrollo Social (SEDESOL) es la entidad mexicana encargada de dar ayuda y alivio a las personas que se encuentran en estado de marginación y/o carencia. Cuenta anualmente con un estimado de 19 programas sociales a nivel federal de los que se desprenden aproximadamente 330 subprogramas dando apoyo a más de 80 millones de mexicanos, lo que representa una covertura superior al 60\% del total de la población mexicana.\\
El mecanismo principal de integración a los programas sociales tiene como parte fundamental la aplicación del Cuestionario Único de Información Socioeconómica (CUIS)\cite{cuis} el cuál es la base mínima de información de los universos de beneficiarios de cada programa. Algunos programas incorporan más preguntas a dicho cuestionario para obtener una mayor información referente a las condiciones específicas que pretenden mejorar. El CUIS es aplicado por hogar, el cual se define como el conjunto de personas que comparten techo, manutención y comida; es decir, en un inmueble pueden convivir diversos hogares si estos no comparten gastos y/o no comen de la misma preparación de alimentos.\\
Debido a que la información asentada en dicho cuestionario es la utilizada para que un programa decida otorgar un beneficio a un hogar, el aliciente para contestarlo de manera falaz es muy elevado pues el encuestado puede declarar ingresos menores o vulnerabilidades falsas con la intención de ser considerado dentro de la definición de población objetivo de cierto programa. Es importante considerar que la mayor parte de los programas no realizan una verificación domiciliar para corroborar las respuestas, pues dicho cuestionario es llenado por medio de una aplicación digital o en un módulo de SEDESOL.\\
De esta manera, es importante contar con un modelo estadístico que nos permita definir qué tan verosímil es que un hogar en partícular se encuentre dentro de las definiciones de vulnerabilidades o carencias. Por lo que este proyecto pretende presentar una metodología para realizar una estimación del ingreso por medio de información que se considere con alta probabilidad de ser verídica como lo es la versión del CUIS del programa PROSPERA\cite{prospera} que realiza verificaciones a toda su población beneficiada.\\
El Sistema de Focalización de Desarrollo (SIFODE) es el encargado de consolidar la información socioeconómica de los hogares a través de los cuestionarios antes descritos, siendo posible que estos hayan sido beneficiarios anteriores de algún programa o no. A partir de los resultados de los cuestionarios el mismo SIFODE es quien está encargado de evaluar los mismos por medio del modelo multidimensional de pobreza dado por el Consejo Nacional de Evaluación de la Política de Desarrollo Social (CONEVAL)\cite{multidimensional}.\\

\section{Carencias Sociales}
SEDESOL considera que la pobreza es causada en base a 6 tipos de carencias sociales, clasificadas de la siguiente manera:\\
\begin{itemize}
    \item Seguridad Social
    \item Salud
    \item Educación
    \item Alimentación
    \item Vivienda
    \item Ingreso
\end{itemize}
Dichas carencias están definidas por CONEVAL\cite{coneval} que a partir de la Ley General de Desarrollo Social (LGDS) aprovada en 2004 donde se establece la promoción de condiciones que aseguren el disfrute de los derechos sociales, así como el impulso de un desarrollo económico con sentido social que eleve el ingreso de la población y contribuya a reducir la desigualdad\ref{carencias}. Para poder garantizar lo anterior se creó el mencionado CONEVAL como un instrumento de evaluación y seguimiento de las políticas de desarrollo social. El mismo aunque es un organismo público, a la vez cuenta con autonomía técnica y de gestión. Entre sus deberes de ejercicio se encuentra establecer los lineamientos y criterios para la definición, identificación y medición de la pobreza.\\
La LGDS establece que la medición de la pobreza efectuada por CONEVAL debe ser efectuada cada 2 años a nivel estatal y cada 5 a nivel municipal, utilizando la información generada por el Instituto Nacional de Estadística y Geografía (INEGI)\cite{inegi} considerando al menos los siguientes indicadores:\\
\begin{itemize}
    \item Ingreso corriente per cápita
    \item Rezago educativo promedio en el hogar
    \item Acceso a los servicios de salud
    \item Acceso a la seguridad social
    \item Calidad y espacios de la vivienda
    \item Acceso a los servicios básicos en la vivienda
    \item Acceso a la alimentación
    \item Grado de cohesión social
\end{itemize}
La necesidad de usar los ocho indicadores implica la generación de una medición multidimensional de la pobreza
