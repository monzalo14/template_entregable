\chapter{Introducción}
La Secretaría de Desarrollo Social opera más de 20 programas a nivel federal. Dichos programas están diseñados con el fin específico de incrementar ciertos derechos sociales de la población objetivo. Algunos de esos programas son de cobertura universal\footnote{Se dice que un programa es de cobertura universal cuando busca atender a prácticamente toda la población alrededor del lugar de operación del programa. Por otro lado, los programas que utilizan estrategias de focalización buscan atender a la población que cumple con ciertas características específicas.}, mientras que otros utilizan distintas estrategias de focalización. Para tener posibilidad de evaluar el impacto y la eficiencia en el gasto social, ambos tipos de programas requieren de una actualización fidedigna de la información con la que cuentan sobre la población atendida. Además, los programas que hacen focalización generalmente requieren de esa información previo a iniciar operaciones. En ambos casos existe un reto difícil de paliar: tanto las posibles beneficiarias como las personas que operan los programas pueden tener incentivos a sub o sobrereportar ciertos aspectos de su información socioeconómica.
\par
\noindent
Este proyecto busca generar una solución tecnológica a este tipo de problemas, mejorando así la calidad de los datos con los que se cuenta para dirigir la política social. Utilizando diversos insumos proveídos tanto por la DGGPB como por instancias externas, nos proponemos a construir modelos para corregir por ese sub y sobrereportaje latente. En este documento se precisan detalles sobre esos insumos de datos y sobre la infraestructura y código utilizados para procesarlas.
