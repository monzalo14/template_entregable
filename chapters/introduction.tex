\chapter{Introducción}
\section{Fuentes de Información}
La Secretaría de Desarrollo Social (SEDESOL) es la entidad mexicana encargada de dar ayuda y alivio a las personas que se encuentran en estado de marginación y/o carencia. Cuenta anualmente con un estimado de 19 programas sociales a nivel federal de los que se desprenden aproximadamente 330 subprogramas dando apoyo a más de 80 millones de mexicanos, lo que representa una covertura superior al 60\% del total de la población mexicana.\\
El mecanismo principal de integración a los programas sociales tiene como parte fundamental la aplicación del Cuestionario Único de Información Socioeconómica (CUIS) el cuál es la base mínima de información de los universos de beneficiarios de cada programa. Algunos programas incorporan más preguntas a dicho cuestionario para obtener una mayor información referente a las condiciones específicas que pretenden mejorar. El CUIS es aplicado por hogar, el cual se define como el conjunto de personas que comparten techo, manutención y comida; es decir, en un inmueble pueden convivir diversos hogares si estos no comparten gastos y/o no comen de la misma preparación de alimentos.\\
Debido a que la información asentada en dicho cuestionario es la utilizada para que un programa decida otorgar un beneficio a un hogar, el aliciente para contestarlo de manera falaz es muy elevado pues el encuestado puede declarar ingresos menores o vulnerabilidades falsas con la intención de ser considerado dentro de la definición de población objetivo de cierto programa. Es importante considerar que la mayor parte de los programas no realizan una verificación domiciliar para corroborar las respuestas, pues dicho cuestionario es llenado por medio de una aplicación digital o en un módulo de SEDESOL.\\
De esta manera, es importante contar con un modelo estadístico que nos permita definir qué tan verosímil es que un hogar en partícular se encuentre dentro de las definiciones de vulnerabilidades o carencias. Por lo que este proyecto pretende presentar una metodología para realizar una estimación del ingreso por medio de información que se considere con alta probabilidad de ser verídica como lo es la versión del CUIS del programa PROSPERA que realiza verificaciones a toda su población beneficiada.\\
