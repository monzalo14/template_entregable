\chapter{Conclusión}
\label{chap:conclusion}
En este proyecto se planteó el problema del reportaje incorrecto como un problema de datos faltantes, modelado a partir de redes bayesianas. Para esto, se tomaron en cuenta dos conjuntos de variables: las variables de los cuestionarios ENCASEH, aplicados por el programa PROSPERA, y las variables que componen el índice de marginación de CONAPO, discretizadas de acuerdo a una partición heurística.
\par
\noindent
Se consideraron distintos conjuntos de variables en el modelado, evaluando primero la estructura resultante a través de la devianza de validación, y después construyendo un clasificador a partir de la distribución a posteriori para las variables verificadas. El clasificador fue evaluado con el área bajo la curva ROC.
\par
\noindent
Una vez elegida la especificación para el modelo, se construyó el código que recibe las respuestas por parte del aplicativo móvil, calcula la distribución a posteriori de las variables dada la evidencia y genera la clasificación que indica si hay o no que verificar ese cuestionario. En el caso de la implementación de este piloto, se decidió verificar todas las observaciones para fines del reentrenamiento del modelo.
